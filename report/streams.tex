
The Stream class is used to represent potentially infinite lists of data.
This is a fairly common concept in functional programming languages but is rare in C++.
Streams can be converted to finite Collections via the \code{take()} function.

Observe the following example, again computing the sum of the numbers 1 to 100.

\begin{lstlisting}
Stream<int> ints = from(1);
Collection<int> ints_1_100 = ints.take(100);
int sum = ints_1_100.reduceLeft([](int x, int y) {return x + y;});
\end{lstlisting}

...or...

\begin{lstlisting}
int sum = from(1).take(100).reduceLeft([](int x, int y) {return x + y;});
\end{lstlisting}

Here, we define the infinite Stream with the \code{from()} function, which by default returns a Stream of elements of its argument's type where each element is incremented by 1.
We then use the \code{take()} function to convert our Stream into a finite Collection.
We then reduce this Collection with an addition function.

Note that Streams can be arbitrarily complex.
Below, we define a Stream to generate the Fibonacci sequence with the help of one of the Stream creation functions provided by the C++ Collections library.


\begin{lstlisting}
auto fibs = recurrence([](std::tuple<int,int> t) {
    return std::get<0>(t) + std::get<1>(t);
}, std::make_tuple(0, 1));

std::cout << fibs.take(5) << std::endl;
// [0,1,1,2,3]
\end{lstlisting}





\subsection{Implementation Details}

The Stream class is a self-referential data structure, meaning that in addition to the head, or the first value in the list, the class stores a pointer to a function that returns another Stream at all times.
Recall the definition of the Fibonacci Stream generator from above.
Written without the macro, the definition is as follows:

\begin{lstlisting}
std::function<Stream<int>(int, int)> fibs = [&](int prev, int curr) -> Stream<int> {
    return Stream<int>(curr, [=]() -> Stream<int> {
        return fibs(curr, prev + curr);
    });
};

Stream<int> tenfibs = fibs(0, 1).take(3);
\end{lstlisting}

The Stream generator \code{fibs}, when called with arguments \code{0} and \code{1}, creates a new Stream that has a head value of 1 and an internal pointer to the function `fibs(1, 1)`.
Then, when we ask to \code{take(3)} elements from the Stream, the structure returns its head, and then lazily computes the remainder of the elements it needs to satisfy our request.
It does this by returning the result of its stored function pointer.
In this case, `tenfibs` would first return 1 becuase that is its head.
It would then return the head of the Stream that results from the evaluation of the function it stores a pointer to, namely \code{fibs(1, 1)}, which is 1.
The result of \code{fibs(1, 1)} also stores a pointer to another function, \code{fibs(1, 2)}.
To find the third and final value, the Stream will evaluate this function and return the resulting Stream's head, which is 2.
For a step-by-step visual depiction of the underlying state of our Stream during the \code{take(3)} function call, see the graphic below:

\begin{lstlisting}
tenfibs.head      tenfibs.gen
     1             fibs(0,1)
     1             fibs(1,1)
     2             fibs(1,2)
     3             fibs(2,3)
\end{lstlisting}

